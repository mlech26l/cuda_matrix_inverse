% !TEX root =  root.tex


\section{PROBLEM STATEMENT}\label{sec:problem}
First we will repeat some mathematical definitions for the sake of completeness.
\begin{definition}
	\textbf{Inverse Matrix}
	Let $A \in \mathbb{R}^n \times \mathbb{R}^n$ and $I_n$ be the identity matrix of dimension $n$, then the matrix $A^{-1} \in \mathbb{R}^n \times \mathbb{R}^n$ is called \textbf{inverse matrix} of $A$ if and only if
	\begin{equation}\label{eq:inverse}
	A \cdot A^{-1} = I_n
	\end{equation}
\end{definition}
A quadratic matrix $A$ is called \emph{regular} if there exists a inverse matrix of $A$. If there exists no inverse of $A$ then $A$ is called \emph{singular}.\\
Inverse matrices can be though to find, but have various of different uses especially in control theory to describe linear systems.  
\vspace{0.2cm}\\
There are a lot of ways to compute the inverse of a matrix, we will introduce some of the most common ones.\\
\subsection*{Gaussian-Elimination}
TODO: Add definition of gauss
\vspace{0.2cm}\\
\subsection*{Inversion by adjugate matrix}
The inverse $A^{-1}$ of $A$ can be computed by:
\begin{equation}
	A^{-1}_{i,j} := \frac{1}{\text{det}(A)} \text{adj}_{i,j}(A)
\end{equation}
with
\begin{definition}
	\textbf{Adjugate matrix} Let $M_{i,j}$ be the matrix $A$ where the $i$-th row and the $j$-th column are removed. Then:
	\[ \text{\textnormal{adj}}_{i,j}(A) = (-1)^{i+j} \text{\textnormal{det}}(M_{j,i}) \]
\end{definition}
This way of inverting a matrix works well for small $n$ (especially if $n \leq 3$), because the determinant is easy to compute in these cases.\\

\subsection*{CUDA terminology}
In the following the term \emph{host} refers to program code that is executed on the CPU. On the other hand is the term \emph{device} used to refer to program code that is executed on the GPU or computation card.\\
A \emph{kernel} is a function that can be executed on the device.

