% !TEX root =  root.tex


\section{METHODS}
\label{sec:methods}
\subsection*{Verification}
To verify if a matrix $A$ was inverted correctly one has to check if equation (\ref{eq:inverse}) holds. So our implementation also requires functions to multiply two matrices and check if a matrix is equal to the identity matrix. Ideally these procedures are also done using CUDA.\\
Lastly we need test matrices to run our implementation. One solution might be to read them from a file, another would be to generate the matrices randomly. We choose the latest because this way we can make use of CUDA again, in form of the cuRAND \footnote{\url{https://developer.nvidia.com/curand}} library for random number generation. So all in all our implementation will consists of the steps listed in figure (\ref{fig:steps})

\begin{figure}
\begin{tikzpicture}
	\node (entry) at (0,-0.5) [draw=none,rectangle]  {};
	\node (curand) at (0,-2) [draw,rectangle,fill=blue!10,align=center,minimum width=5cm, minimum height=1.2cm] {Generate random matrix\\using cuRAND};
	\node (inversion) at (0,-4) [draw,rectangle,fill=blue!10,align=center,minimum width=5cm, minimum height=1.2cm] {Invert matrix};
	\node (multiply) at (0,-6) [draw,rectangle,fill=blue!10,align=center,minimum width=5cm, minimum height=1.2cm] {Multiply $A$ with $A^{-1}$\\using shared memory algorithm};
	\node (check) at (0,-8) [draw,rectangle,fill=blue!10,align=center,minimum width=5cm, minimum height=1.2cm] {Check if equal to identity matrix\\using reduction} ;
	\draw[-triangle 60] (entry) to (curand);
	\draw[-triangle 60] (curand) to (inversion);
	\draw[-triangle 60] (inversion) to (multiply);
	\draw[-triangle 60] (multiply) to (check);
	
\end{tikzpicture}
\centering
\label{fig:steps}
\caption{Steps done by our implementation}
\end{figure}



