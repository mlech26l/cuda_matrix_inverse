% !TEX root =  root.tex


\section{METHODS}
\label{sec:methods}
\subsection*{Verification}
To verify if a matrix $A$ was inverted correctly one has to check if equation (\ref{eq:inverse}) holds. So our implementation also requires functions to multiply two matrices and check if a matrix is equal to the identity matrix. Ideally these procedures are also done using CUDA.\\
Lastly we need test matrices to run our implementation. One solution might be to read them from a file, another would be to generate the matrices randomly. We choose the latest because this way we can make use of CUDA again, in form of the cuRAND \footnote{\url{https://developer.nvidia.com/curand}} library for random number generation. So all in all our implementation will consists of the steps listed in figure (\ref{fig:steps})

\begin{figure}
\begin{tikzpicture}
	\node (entry) at (0,-0.5) [draw=none,rectangle]  {};
	\node (curand) at (0,-2) [draw,rectangle,fill=blue!10,align=center,minimum width=5cm, minimum height=1.2cm] {Generate random matrix\\using cuRAND};
	\node (inversion) at (0,-4) [draw,rectangle,fill=blue!10,align=center,minimum width=5cm, minimum height=1.2cm] {Invert matrix};
	\node (multiply) at (0,-6) [draw,rectangle,fill=blue!10,align=center,minimum width=5cm, minimum height=1.2cm] {Multiply $A$ with $A^{-1}$\\using shared memory algorithm};
	\node (check) at (0,-8) [draw,rectangle,fill=blue!10,align=center,minimum width=5cm, minimum height=1.2cm] {Check if equal to identity matrix\\using reduction} ;
	\draw[-triangle 60] (entry) to (curand);
	\draw[-triangle 60] (curand) to (inversion);
	\draw[-triangle 60] (inversion) to (multiply);
	\draw[-triangle 60] (multiply) to (check);
	
\end{tikzpicture}
\centering
\label{fig:steps}
\caption{Steps done by our implementation}
\end{figure}
\subsection*{Random matrix generation}
The cuRand library for generating pseudo and quasi random numbers consists of two major modules:
the host API and the device API.\\
The device API, which can be accessed by \texttt{curand\_kernel.h}, are low level functions that can only be called from within a kernel. \\
The host API on the other hand, can be access by \texttt{curand.h} and contains high level functions for host code. With the host API, entire arrays can be filled with random numbers using only one API call. \\
Because a row-major-ordering matrix of dimension $n$ by $n$ is represented as an array of size $n^2$, a random matrix can be generated with only one API call. 

\subsection*{Matrix multiplication using shared memory}

\subsection*{Checking if matrix is identity matrix}
We need a procedure which checks if a given matrix $A$ of dimension $n$-by-$n$ is equal to the identity matrix $I_n$. Formally defined:
\begin{equation*}
is\_id(A) := \begin{cases}
1 & \text{if } A \text{ has 1's in the diagonal, and 0's elsewhere}\\
0 & \text{otherwise}
\end{cases}
\end{equation*}
So $is\_id$ maps a matrix of $\mathbb{R}^{n \times n}$ to a single value $\{0,1 \}$. Therefore a implementation of $is\_id$ can be done efficiently using a \textbf{reduction}.  \\
We adapted the code from the CUDA 7.5 samples directory, namely from \texttt{/samples/6\_Advanced/reduction/}. The sample code is based on the whitepaper from Harris \cite{Harris2010} and optimizes the reduction for a summation of an array.